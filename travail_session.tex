\documentclass[12pt]{article}
\usepackage[utf8]{inputenc}
\usepackage{color,dcolumn,graphicx,hyperref}
\usepackage{wrapfig}
\usepackage{natbib}

\begin{document}



\title{Travail de session}

\author{Josiane Côté Audet, Gabriel Boilard, Kathryne Moreau et Élyse Paquette}
\date{23 avril 2019}

\maketitle
\section{Résumé}
\section{Introduction}

Un réseau écologique est le réseau de liens entre les différents individus ou espèces d’un écosystème. Ces liens peuvent représenter une relation antagoniste (prédateur-proie) ou mutualiste (pollinisateur-plante). Ceci implique qu’un changement sur un espèce peut aller affecter plusieurs autres espèces interrelié. Par exemple, l'élimination de plantes exotiques peut entraîner une pollinisation réduite d'une plante indigène rare par des changements dans la population de pollinisateurs qui se nourrissent à la fois de plantes indigènes et exotiques.\citep{moore1978interspecific} De plus, les liens dans le réseau d’interactions écologiques peuvent être faible ou fort. Plusieurs études ont démontré que les réseaux sont composés de quelques interactions forts au sein d’une matrice d’interactions faibles. Le degré (ou ‘connectivité’) de ces interactions sont évalué par le nombre de liens impliquant chaque espèce. La diversité des interactions est directement proportionnelle à la diversité d’espèce présente dans un réseau. Elle est généralement mesurée par rapport au nombre d’espèce dans le réseau. Ces réseaux écologiques prennent de plus en plus d’expansion dans le monde de la science. Il est  difficile d’ignorer l’impact d’une activité anthropique sur un  réseau lorsque ses conséquences sont étudiées.** Le but de l’étude est de déterminer si le réseau de collaboration entre des étudiants du même programme universitaire est différent des réseaux écologiques. La question a été séparé en deux parties. Après avoir fait le réseau des interactions des étudiants, nous avons déterminer s’il y avait une corrélation entre le nombre de liens par étudiant et leur sexe. Nous avons ensuite déterminé s’il y avait une corrélation entre la région d’origine et le lien entre deux étudiants.\citep{curry1989geographic}

\section{Méthode et Résultats}

\begin{figure}
\includegraphics[width=200]{graph_region.pdf}
\end{figure}

\begin{figure}
\includegraphics[width=200]{graph_reseau.pdf}
\end{figure}

\begin{figure}
\includegraphics[width=200]{graph_reseau_classe.pdf}
\end{figure}

\begin{figure}
\includegraphics[width=200]{graph_sexe.pdf}
\end{figure}

\input{tableau_diete.tex}

\section{Discussion}


\section{Références}

\bibliographystyle{apalike}
\bibliography{references}



\end{document}
